\begin{frame}{The $\Upupsilon$-Meson}
 
	\begin{minipage}[c][.2\textheight]{.8\textwidth}
	 	\begin{itemize}
			\item bound state of \textbf{b$\bar{\z{b}}$}\vspace*{10pt}
			\item decay channels:
		\end{itemize}
	\end{minipage}
	\begin{minipage}{.18\textwidth}
		\fig{T1}{.2}
	\end{minipage}
	
	\begin{figure}
		\subfig[.37]{diag2}{.25}{\SI{82}{\%}}
		\subfig[.26]{diag1}{.25}{\SI{9}{\%}}
		\subfig[.35]{diag3}{.25}{\SI{2}{\%}}
	\end{figure}
	\begin{itemize}\itemfill
		\item mostly decay into gluons which hadronise \ra signals mostly caused by hadrons
		\item leptonic decay splits up into \SIrange{2.5}{3}{\%} for each e, $\upmu$ and $\uptau$
	\end{itemize}

\end{frame}

%%%%%%%%%%%%%%%%%%%%%%%% FRAME 1 %%%%%%%%%%%%%%%%%%%%%%%%%%%%%%%
\subsection{Discovery}
\begin{frame}{Discovery Paper}

	\fig{BDis}{.84}
	
\end{frame}

%%%%%%%%%%%%%%%%%%%%%%%% FRAME 2 %%%%%%%%%%%%%%%%%%%%%%%%%%%%%%%
\begin{frame}{Setup}
 
	\vspace*{-10pt}
	\only<1>{\fig{BSetup}{.72}}
	\only<2>{\fig{BSetup1}{.72}}
	\only<3>{\fig{BSetup2}{.72}}
	\only<4>{\fig{BSetup3}{.72}}
	\only<5>{\fig{BSetup4}{.72}}
	\only<6>{\fig{BSetup5}{.72}}
	\only<7>{\fig{BSetup6}{.72}}
	\only<8>{\fig{BSetup7}{.72}}
	\only<9>{\fig{BSetup8}{.72}}
	\only<10>{\fig{BSetup9}{.72}}
	
	\begin{minipage}[c][.15\textheight]{\textwidth}
		\begin{itemize}
			\only<1>{\item[] } 
			\only<2>{\item \SI{400}{\giga\electronvolt} proton beam shot on narrow target (Pt/Cu) with \SI{30}{\%} interaction length} 
			\only<3>{\item hadron filter out of Be with 18 interaction length (\SIrange{3}{5}{^{\circ}} horiz. and \SI{\pm.5}{^{\circ}} vert.)}
			\only<4>{\item heavy metal (Steel, W) shielding to minimise particle leakage}
			\only<5>{\item tungsten beam dump} 
			\only<6>{\item additional shielding out of polyethylene and more steel}
			\only<7>{\item spectrometer dipole magnets with horizontal field}
			\only<7>{\item both arms are symmetric to drawing plane and detect $\upmu^{+}$ and $\upmu^{-}$}
			\only<8>{\item scintillation hodometers and wire chambers for tracking (limit of \SI{10e7}{counts\per s})}
			\only<9>{\item solid iron magnet to partially refocus and redetermine muon momentum}
			\only<10>{\item \v{C}erenkov counter to prevent low momentum muon triggers}
		\end{itemize}
	\end{minipage}

\end{frame}
%%%%%%%%%%%%%%%%%%%%%%%% FRAME 3 %%%%%%%%%%%%%%%%%%%%%%%%%%%%%%%
\begin{frame}{Intermezzo: Sideband Fit}
 
	\vspace*{-10pt}
	\only<1>{\hspace*{-20pt}\figp{SB1}{.60}}
	\only<2>{\hspace*{-20pt}\figp{SB2}{.60}}
	\only<3>{\hspace*{-20pt}\figp{SB3}{.60}}
	\only<4>{\hspace*{-20pt}\figp{SB4}{.60}}
	\only<5>{\hspace*{-20pt}\figp{SB5}{.60}}
	\only<6>{\hspace*{-20pt}\figp{SB6}{.60}}
	
	\begin{minipage}[c][.15\textheight]{\textwidth}
		\begin{itemize}
			\only<1>{\item typical shape of data in particle physics: continuous background with a small bump} 
			\only<2>{\item background extraction with a fit of the whole set does not work well} 
			\only<3>{\item divide the data set in a signal and two background parts (\ra side bands)}
			\only<4>{\item fitting just the side bands yields a much better result}
			\only<5>{\item getting the signal distribution by subtracting the full data set by the fit} 
			\only<6>{\item get width and position of the signal by a fit}
		\end{itemize}
	\end{minipage}

\end{frame}
%%%%%%%%%%%%%%%%%%%%%%%% FRAME 4 %%%%%%%%%%%%%%%%%%%%%%%%%%%%%%%
\begin{frame}{Results}
 
	\begin{minipage}[c][.8\textheight]{.78\textwidth}
		\begin{itemize}
			\itemfill
			\item statistically significant enhancement at \SI{9.5}{\giga\electronvolt} $\upmu^{+}\upmu^{-}$ mass
			\item solid line background fit using side band method
			\begin{equation*} \dfrac{\z{d}^2\upsigma}{\z{dmdy}} = \z{A}\mathrm{e}^{-\z{bm}} \end{equation*}
			\item fit expects 350 events in excluded region but 770 events in data
			\item bump wider than resolution of detector (FWHM: \SI{.5\pm.1}{\giga\electronvolt})
			\item simple Gaussian fit of background subtracted data yields:
			{\usebeamercolor[fg]{title}\begin{align*}
				\z{\textbf{m}} &= \textbf{\SI{9.54\pm.04}{\giga\electronvolt}}\\
				\z{\textbf{FWHM}} &= \textbf{\SI{1.16\pm.09}{\giga\electronvolt}}
			\end{align*}}
			\item same goodness of fit with two Gaussians with fixed with of detector resolution \ra later $\Upupsilon$ and $\Upupsilon'$

		\end{itemize}

	\end{minipage}
	\begin{minipage}{.2\textwidth}
		\figc{BRes1}{.69}{\scriptsize{dimuon production cross section as a function of invariant mass}}
	\end{minipage}

	
\end{frame}

%%%%%%%%%%%%%%%%%%%%%%%% FRAME 5 %%%%%%%%%%%%%%%%%%%%%%%%%%%%%%%
\begin{frame}{Verification}

	\begin{itemize}\itemfill
		\item same sign dimuon spectrum ($\upmu^{+}\upmu^{+}$ and $\upmu^{-}\upmu^{-}$)
		\begin{itemize}
			\item upper limit on combined effects of accidental coincidences and hadronic decays
		\end{itemize}
		\item re-measurement of the muon momentum by the second magnet and by PCW at the centre of the first magnet
		\begin{itemize}
			\item avoid misidentified \ch{\ensuremath{\uppsi}->\ensuremath{\upmu^+}+\ensuremath{\upmu^-}} at high mass
			\item confirmed by clear separation of the $\uppsi$ and $\uppsi'$ in the Figure
		\end{itemize}
	\end{itemize}
	
	\begin{minipage}[c][.35\textheight]{0.58\textwidth}
		\begin{itemize}\itemfill
			\item study of various subsets of the data with different magnetic fields
			\begin{itemize}
				\item check for apparatus bias
			\end{itemize}
			\item study of data with and without target
			\begin{itemize}
				\item rule out signal created from beam dump
			\end{itemize}
		\end{itemize}
	\end{minipage}
	\begin{minipage}{.4\textwidth}
		\fig{BRes}{.4}
	\end{minipage}\vspace*{-20pt}

	
\end{frame}

%%%%%%%%%%%%%%%%%%%%%%%% FRAME 6 %%%%%%%%%%%%%%%%%%%%%%%%%%%%%%%
\subsection{Further Experiments}
\begin{frame}{PLUTO Collaboration Paper}

	\fig{BPluto}{.84}
	
\end{frame}

%%%%%%%%%%%%%%%%%%%%%%%% FRAME 7 %%%%%%%%%%%%%%%%%%%%%%%%%%%%%%%
\begin{frame}{PLUTO Detector}
 
	\vspace*{-10pt}
	\only<1>{\fig{PlutoDet}{.72}}
	\only<2>{\fig{PlutoDet5}{.72}}
	\only<3>{\fig{PlutoDet1}{.72}}
	\only<4>{\fig{PlutoDet2}{.72}}
	\only<5>{\fig{PlutoDet3}{.72}}
	\only<6>{\fig{PlutoDet4}{.72}}
	
	\begin{minipage}[c][.15\textheight]{\textwidth}
		\begin{itemize}
			\only<1>{\item[]  } 
			\only<2>{\item up to \SI{5}{\giga\electronvolt} electron beams collided in the centre} 
			\only<3>{\item cylindrical proportional wire chamber with \SI{92}{\%} coverage}
			\only<3>{\item tracking and momentum measurement}
			\only<4>{\item cylindrical array of shower counters with \SI{8.6}{radiation length} and \SI{94}{\%} coverage}
			\only<4>{\item calorimeter to measure the full energy}
			\only<5>{\item magnet with \SI{1.69}{\tesla}}
			\only<6>{\item muon chambers}
		\end{itemize}
	\end{minipage}

\end{frame}
%%%%%%%%%%%%%%%%%%%%%%%% FRAME 8 %%%%%%%%%%%%%%%%%%%%%%%%%%%%%%%
\begin{frame}{Results}
	
	\begin{minipage}[c][.4\textheight]{0.65\textwidth}
		\begin{itemize}\itemfill
			\item acquiring cross section by scanning $\sqrt{s}$ in steps of either \SI{5}{\giga\electronvolt} or \SI{10}{\giga\electronvolt}
			\item only considering hadronic decay products
			\item remove 1/s background
			\item reduce cosmic ray background by use of bunch crossing time (bunched beam structure vs continuous background)
		\end{itemize}
	\end{minipage}
	\begin{minipage}{.31\textwidth}
		\figc{BPlutoRes}{.3}{\footnotesize{total cross section for hadron production}}
	\end{minipage}
	
	\begin{itemize}\itemfill
		\item remove QED events with coplanarity cut and shower recognition
		\item remove beam gas interaction by cuts on visible energy and missing mass
		\item fitting data with Gaussian yields:
		{\usebeamercolor[fg]{title}\begin{align*}
			\z{\textbf{m}} &= \textbf{\SI{9.46\pm.01}{\giga\electronvolt}}\\
			\mathbf{\upsigma_{\z{Gauss}}} &= \textbf{\SI{7.8\pm.9}{\mega\electronvolt}}
		\end{align*}}
	\end{itemize}\vspace*{-20pt}

	
\end{frame}
%%%%%%%%%%%%%%%%%%%%%%%% FRAME 9 %%%%%%%%%%%%%%%%%%%%%%%%%%%%%%%
\begin{frame}{Charge}
	
	\begin{itemize}\itemfill
		\item relation of total hadronic cross section to resonance mass and the resonance width
		\begin{equation*} \int\upsigma_{\z{h}}\z{dM} = \frac{6\uppi^{2}}{{\z{M}_{\z{R}}}^{2}}\frac{\Upgamma_{\z{ee}}\Upgamma_{\z{h}}}{\Upgamma_{\z{tot}}} \end{equation*}
		\item standard assumption: $\Upgamma_{\z{tot}}\approx\Upgamma_{\z{h}}$ \ra direct measurement of $\Upgamma_{\z{ee}}$
		{\usebeamercolor[fg]{title}\begin{equation*} \Upgamma_{\z{ee}} = \SI{1.3\pm.4}{\kilo\electronvolt} \end{equation*}}\vspace*{-20pt}
		\item theoretical predictions from the same model:
		\begin{center}
			\begin{tabular}{c|c|c}
						\multirow{2}{*}{Decay Mode}	& \multicolumn{2}{c}{Width [\SI{}{\kilo\electronvolt}]}\\\cline{2-3}
													& $\z{e} = \pm\sfrac{1}{3}$	& $\z{e} = \pm\sfrac{2}{3}$\\\hline
						$\upmu^{+}\upmu^{-}$ 		& \SI{.70\pm.09}{} 	& \SI{2.8\pm.4}{}
			\end{tabular}		             
		\end{center}\vspace*{10pt}
		\item \usebeamercolor[fg]{title}\textbf{implies charge of -\sfrac{1}{3} for the bottom quark}

	\end{itemize}

	
\end{frame}